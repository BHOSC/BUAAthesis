% !Mode:: "TeX:UTF-8"
\chapter{使用说明}
    \section{基本范例}
    \begin{table}[!ht]
    \begin{center}
        \begin{tabular}{|c|c|}
        \hline
        本科生论文基本结构 & 研究生论文基本结构\\\hline\hline
        封面 & 封面(中、英文)\\
        扉页 & 题名页、独创性声明和使用授权书\\
        中英文摘要 & 中英文摘要\\
        目录 & 目录\\
        正文 & 图表清单及主要符号表(根据情况可省略)\\
        致谢 & 主体部分\\
        参考文献 & 参考文献\\
        附录 & 附录\\
        ~~ & 攻读硕士/博士期间取得的研究\slash 学术成果\\
        ~~ & 致谢\\
        ~~ & 作者简介\\
        \hline
        \end{tabular}
        \end{center}
    \end{table}
\qquad 本科生论文结构推荐按如下的代码形式来组织整个论文。具体各个参数及部分请见第xx节。\par
\begin{lstlisting}[language={LaTeX}]
% !Mode:: "TeX:UTF-8"
\documentclass[bachelor,openany,oneside,adobefonts]{buaathesis}
\begin{document}

% 用户信息
% !Mode:: "TeX:UTF-8"

% 作者中英文名
\thesisauthor
{姓名}
{Name}

% 论文中英文标题
\thesistitle
{基于Texlive的北航毕设论文模板设计}
{How to design the BUAA-thesis with \LaTeX{}}

% 学院中英文名,中文不需要“学院”二字
% 院系英文名可参见
% http://ev.buaa.edu.cn/education/index.php?page=department
\school
{XXX}
{School of XXX}

% 专业中英文名,中文不需要“专业”二字
\major
{XXXX}
{XXXX Engineering}

% 导师中英文名
\teacher
{导师中文名}
{Name of tutor}

% 中图分类号,可在 http://www.ztflh.com/ 查询
\category{TP312}

% 本科生为毕设开始时间;研究生为学习开始时间
\thesisbegin{2011}{09}{01}

% 本科生为毕设结束时间;研究生为学习结束时间
\thesisend{2012}{07}{01}

% 毕设答辩时间
\defense{2012}{06}{01}

% 中文关键字
\ckeyword{北航开源俱乐部,\LaTeX{},论文}

% 英文关键字
\ekeyword{BHOSC,\LaTeX{},Thesis}
 % 公共信息部分
% !Mode:: "TeX:UTF-8"

% 班级
\class{XXXX}

% 学号
\studentID{XXXXXXXX}

% 单位代码
\unicode{10006}

% 论文时间,用于首页
\thesisdate{2012}{6}
 % 本科生信息部分
% 任务书信息
% !Mode:: "TeX:UTF-8"
% 任务书中的信息
%% 原始资料及设计要求
\assignReq
{原始资料及设计要求第一行}
{原始资料及设计要求第二行}
{原始资料及设计要求第三行}
{原始资料及设计要求第四行}
{原始资料及设计要求第五行}
%% 工作内容
\assignWork
{工作内容第一行}
{工作内容第二行}
{工作内容第三行}
{工作内容第四行}
{工作内容第五行}
{工作内容第六行}
%% 参考文献
\assignRef
{参考文献第一行}
{参考文献第二行}
{参考文献第三行}
{参考文献第四行}
{参考文献第五行}
{参考文献第六行}
{参考文献第七行}
{参考文献第八行}


% 封面 & 任务书 & 声明
\maketitle
% 摘要
% !Mode:: "TeX:UTF-8"
\begin{cabstract}
这里是中文摘要部分。加长到一行点来看看效果是怎样的,所以这句话其实只是我用来
测试用的,觉得不爽的话就直接删除吧,哇哈哈~\par
这里是另一段的开始。
\end{cabstract}
\begin{eabstract}
Here is the Abstract in English. And this is a test sentence, just for
a test to see how the buaathesis works. You can just ignore this.\par
This is another pargraph.
\end{eabstract}

% 目录
\tableofcontents
% 正文页码样式
\mainmatter

% 正文
\include{file/chapter1}	
\include{file/chapter2}
\include{file/chapter3}	

% 致谢
% !Mode:: "TeX:UTF-8"
\chapter*{致谢}
\addcontentsline{toc}{chapter}{致谢}
感谢国家
\cleardoublepage

% 参考文献
% !Mode:: "TeX:UTF-8"
\nocite{*}
\bibliography{file/bibs}
\addcontentsline{toc}{chapter}{参考文献}
\cleardoublepage


% 附录
\appendix
\include{file/appendix1}
\include{file/appendix2}
\end{document}
\end{lstlisting}
\qquad 研究生则推荐使用如下的代码形式来组织论文。\par
\begin{lstlisting}[language={LaTeX}]
% !Mode:: "TeX:UTF-8"
\documentclass[master,openright,adobefonts]{buaathesis}
\begin{document}

% 用户信息
% !Mode:: "TeX:UTF-8"

% 作者中英文名
\thesisauthor
{姓名}
{Name}

% 论文中英文标题
\thesistitle
{基于Texlive的北航毕设论文模板设计}
{How to design the BUAA-thesis with \LaTeX{}}

% 学院中英文名,中文不需要“学院”二字
% 院系英文名可参见
% http://ev.buaa.edu.cn/education/index.php?page=department
\school
{XXX}
{School of XXX}

% 专业中英文名,中文不需要“专业”二字
\major
{XXXX}
{XXXX Engineering}

% 导师中英文名
\teacher
{导师中文名}
{Name of tutor}

% 中图分类号,可在 http://www.ztflh.com/ 查询
\category{TP312}

% 本科生为毕设开始时间;研究生为学习开始时间
\thesisbegin{2011}{09}{01}

% 本科生为毕设结束时间;研究生为学习结束时间
\thesisend{2012}{07}{01}

% 毕设答辩时间
\defense{2012}{06}{01}

% 中文关键字
\ckeyword{北航开源俱乐部,\LaTeX{},论文}

% 英文关键字
\ekeyword{BHOSC,\LaTeX{},Thesis}
 % 公共信息部分
% !Mode:: "TeX:UTF-8"

% 研究方向
\direction{搬砖}

% 教师职称中英文
\teacherdegree{教授}{Prof.}

% 保密等级
\mibao{机密}

% 论文编号,由10006+学号组成
\thesisID{10006SY0000000}

% 论文提交时间
\commit{2012}{3}{3}

% 学位授予日期
\award{2012}{4}{4}
 % 研究生信息部分

% 中英封面 & 题名页 & 独创声明和使用授权
\maketitle
% 摘要
% !Mode:: "TeX:UTF-8"
\begin{cabstract}
这里是中文摘要部分。加长到一行点来看看效果是怎样的,所以这句话其实只是我用来
测试用的,觉得不爽的话就直接删除吧,哇哈哈~\par
这里是另一段的开始。
\end{cabstract}
\begin{eabstract}
Here is the Abstract in English. And this is a test sentence, just for
a test to see how the buaathesis works. You can just ignore this.\par
This is another pargraph.
\end{eabstract}

% 目录 & 插图目录 & 表格目录
\tableofcontents
\listoffigures
\listoftables
% 符号表
% !Mode:: "TeX:UTF-8"
\begin{denotation}

\item[HPC] 高性能计算 (High Performance Computing)
\item[cluster] 集群
\item[Itanium] 安腾
\item[SMP] 对称多处理
\item[API] 应用程序编程接口
\item[PI]	聚酰亚胺
\item[MPI]	聚酰亚胺模型化合物,N-苯基邻苯酰亚胺
\item[PBI]	聚苯并咪唑
\item[MPBI]	聚苯并咪唑模型化合物,N-苯基苯并咪唑
\item[PY]	聚吡咙
\item[PMDA-BDA]	均苯四酸二酐与联苯四胺合成的聚吡咙薄膜
\item[$\Delta G$]  	活化自由能~(Activation Free Energy)
\item [$\chi$] 传输系数~(Transmission Coefficient)
\item[$E$] 能量
\item[$m$] 质量
\item[$c$] 光速
\item[$P$] 概率
\item[$T$] 时间
\item[$v$] 速度
\end{denotation}


% 正文页码样式
\mainmatter

% 正文
\include{file/chapter1}	
\include{file/chapter2}
\include{file/chapter3}

% 参考文献
% !Mode:: "TeX:UTF-8"
\nocite{*}
\bibliography{file/bibs}
\addcontentsline{toc}{chapter}{参考文献}
\cleardoublepage


% 附录
\appendix
\include{file/appendix1}
\include{file/appendix2}

% 附页标题样式
\backmatter

% 附页
% !Mode:: "TeX:UTF-8"
\chapter{攻读博士/硕士学位期间取得的学术成果}

% !Mode:: "TeX:UTF-8"
\chapter{致谢}

% !Mode:: "TeX:UTF-8"
\chapter{作者简介}

\end{document}
\end{lstlisting}
    
    \section{模板选项}
        \subsection{学位选项}
        \begin{description}
            \item{bachelor} 学士学位;
            \item{master} 学术硕士学位(默认);
            \item{engineer} 专业硕士学位;
            \item{doctor} 博士学位。
        \end{description}

        \subsection{其他选项}
        \begin{description}
            \item{oneside\slash twoside} 单面\slash 双面(默认)打印;
            \item{openany\slash openright} 新的章节在任何页面开始\slash 新的章节从奇数页开始(默认);
            \item{nocolor} 所有链接文字、关键字均为黑色并使用框线表示(锚点为红色方框,超链接为淡蓝色方框,打印时方框将不会被打印),默认关闭。
        \end{description}

    \section{封面及正文前的一些设置}
        \subsection{封面}
            本科生论文封面直接使用\texttt{\textbackslash maketitle}命令,将编译生成论文封面和任务书(任务书中的各项需要自己在assign.tex中填写),以及“本人声明”页。只需将\texttt{userinfo.tex}中的信息填写完整即可自行编译生成。\par
            研究生(包括博士研究生)的毕设论文封面使用\texttt{\textbackslash maketitle}将生成中英文封面、题名页、和独创性声明与使用授权书。只需将\texttt{userinfo.tex}中的信息填写完整即可自行编译生成。
        \subsection{中英文摘要}
            本科生和研究生的论文中英文摘要为\texttt{abstract.tex},请直接按照模板示例进行更改替换即可,关键词以及其他的一些个人论文信息在\texttt{userinfo.tex}中自行定义。
        \subsection{目录}
            生成目录为命令\texttt{\textbackslash tableofcontents},需要xelatex两遍才能正确生成目录。\par
            对于研究生,论文还需要有图表目录以及论文主要符号表。分别使用命令\texttt{\textbackslash listoffigures}\texttt{\textbackslash listoftables},而主要符号表则在\texttt{file/master/denotation.tex}中,请自行按照模板给出的样式替换即可。
    \section{正文}
        \subsection{章节}
            正文中的各个章节,推荐将其每一章分为单独的\texttt{.tex}文件,然后使用\texttt{\textbackslash include\{chapter.tex\}}将其包含进来即可。\par
            章节中的内容如何编写,请见\hyperref[chapter-yufa]{第5章~~\LaTeX{}语法}。
        \subsection{参考文献}
            参考文献使用\texttt{BiBTeX}工具,参考文献的数据库为\texttt{bibs.bib},可以使用记事本等文本编辑器进行编辑。具体如何进行编辑也可参照示例模板给出的范例来编写。在Winedt软件中有具体的增加参考文献的选项;在\url{book.google.com}中搜索到的书籍,在页面的最下方也有\texttt{BiBTeX}的导出选项。\par
            \texttt{.bib}参考文献数据库文件中,每个类别后的第一个为标号,在示例的\texttt{bibs.bib}中第一个书箱的标号为\textbf{lamport1994latex},在引用此文献时,使用\textbf{\textbackslash upcite\{lamport1994latex\}}即可得到此文献\upcite{lamport1994latex}的引用\footnote{此处即为引用了此文献,得到相应的上角标。}。
    \section{正文之后的内容}
        \subsection{附录}
            附录和正文中的章节编写方式一样。无特殊之处。
        \subsection{攻读硕士\slash 博士期间所取得的研究\slash 学术成果(研究生)}
        \subsection{致谢}
        \subsection{作者简介(研究生)}