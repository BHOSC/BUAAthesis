% !Mode:: "TeX:UTF-8"
\chapter{论文的参考文献}
\label{chapter-reference}

注意:研究生学位论文参考文献著录及标引按照国家标准《文后参考文献
著录规则》(GB7714)和中国博硕士学位论文编写与交换格式。

本科毕业论文参考文献的著录均应符合国家有关标准(按GB7714-87《文后
参考文献著录格式》执行)。

本《北航毕业论文\latex{}模板》中参考文献模板
借用\href{http://bbs.ctex.org/forum.php}{中文CTEX社区}中由上海财经大学
的吴凯发布的\href{http://bbs.ctex.org/forum.php?mod=viewthread&tid=33591}
{ GBT7714-2005.bst version1 Beta版},其具体的使用方法及代码思想
等请自行查阅其说明文档。

\section{BibTeX简介}
\label{section-bibtex}
1985年,Oren Patashnik和Lamport开发了BibTeX,其详细使用参阅Nicolas Markey的
《Tame the BeaST: The B to X of BibTeX》。

BibTeX把参考文献的数据放在一个.bib文件中,显示格式放在.bst文件中,
普通用户一般不需要改动.bst文件,只需要维护.bib参考文献数据库即可。

一个.bib文件可以包含多个参考文献条目(entry),每个条目有类型、关键字,
以及题目、作者、年份等字段。常用条目类型有article、book、conference、manual等。
每种类型都有一些自己的规定字段和可选字段,字段间
用逗号分隔。数据库中每个条目的关键字(Bibtexkey)要保持唯一,因为
引用时需要用到此字段作为此条目的标识。

前文中提到含有交叉引用的文档需要编译两遍(当然也可能没提到)。
含有参考文献的文档则更“麻烦”,它需要
依次执行latex、bibtex、latex、latex四次编译操作。
\begin{enumerate}
\item 第一遍latex只把条目的关键字写到中间文件.aux文件中去。
\item bibtex根据.aux、.bib、.bst生成一个.bbl文件,即参考文献列表,它的内容就是thebibliography环境
和一些\verb|\bibetm| 命令。
\item 第二遍latex把交叉引用写到.aux中去。
\item 第三遍latex则在正文中正确地显示引用。
\end{enumerate}

\section{使用JabRef来管理参考文献}
\subsection{JabRef简介}
大多只有使用\latex{}撰写科技论文的研究人员才能完全领略到JabRef的妙不可言,
但随着对Word写作平台上BibTeX4Word插件的开发和便利应用,使用Word撰写文章
且用JabRef推送参考文献同样令人十分愉悦。作为新生代的文献的送和管理
工具(2005年开发),只有不足8M的JabRef(安装版/免安装版)不仅功能齐全、
各种操作也考虑周到,实现科技研究人员在跨操作平台和不同写作环境下终身使用
一个“自己的文献库”不再是一个奢望。

JabRef主页和下载地址:\url{http://jabref.sourceforge.net}。其他更多的介绍可以参考文章\url{http://blog.lehu.shu.edu.cn/shuishousong/A220047.html}。
\subsection{如何管理参考文献}

\section{参考文献条目要求}

\section{参考文献示例}
A.1 普通图书

\upcite{gxzzzzqlyt1993},
\upcite{jiangyouxu1998},
\upcite{tangxujun1999},
\upcite{zhaokaihua1995},
\upcite{wangang1912},
\upcite{zhaoyaodong1998},
\upcite{crawfprd1995},
\upcite{iflai1977},
\upcite{obrien1994},
\upcite{rood2001},
\upcite{angwen1988}。

A.2 论文集、会议录

\upcite{zglxxh1990},
\upcite{ROSENTHALL1963},
\upcite{GANZHA2000}。

A.3 科技报告

\upcite{dtha1990},
\upcite{who1970}。

A.4 学位论文

\upcite{ZHANGZHIXIANG1998},
\upcite{CALMS1965}。

A.5 专利文献

\upcite{LIUJIALIN1993},
\upcite{hblz2001},
\upcite{KOSEKI2002}。

A.6 专著中析出的文献

\upcite{baishunong1998},
\upcite{gjbzjxxflbmyjs1988},
\upcite{hanjiren1985},
\upcite{BUSECK1980},
\upcite{FOURNEY1971},
\upcite{feilisheng1981},
\upcite{MARTIN1996}。

A.7 期刊中析出的文献

\upcite{libingmu2000},
\upcite{taorengji1984},
\upcite{yzdztbmz1978},
\upcite{MARAIS1992},
\upcite{HEWITT1984}。

A.8 报纸中析出的文献

\upcite{Dingwenxiang2000},
\upcite{Zhangtianqing2000}。

A.9 电子文献

\upcite{jiangxiangdong1999},
\upcite{xiaoniu2001},
\upcite{CHRISTINE1998},
\upcite{METCALF1995},
\upcite{TURCOTTE1992},
\upcite{Scitor1983}。

附加测试

B1.连续出版物4.3

\upcite{zgdzxh1936},
\upcite{zgtsgxh1957},
\upcite{AAAS1883}。

顺序编码制数字的压缩性测试

\upcite{angwen1988,baishunong1998,Dingwenxiang2000,gxzzzzqlyt1993,
jiangyouxu1998,tangxujun1999,zhaokaihua1995,wangang1912,zhaoyaodong1998},
\upcite{crawfprd1995},
\upcite{iflai1977},
\upcite{obrien1994},
\upcite{rood2001},
\upcite{angwen1988}。

B2. CAJ-CD B/T 1-2005

中国学术期刊(光盘版)检索与评价数据规范

Data Norm for Retrieval and Evaluation of Chinese Academic Journal(CD) (修订版试行稿)

CAJ-CD

中国学术期刊(光盘版)技术规范CAJ-CD B/T 1-2005

代替CAJ-CD B/T 1-1998

(注意:如下是使用\verb|\cite{}|或\verb|\citet{}|的效果,其区别在于\verb|\cite{}|将只引用某个条目
的编号,而\verb|\citet{}|则会在\verb|\cite{}|的基础上加上作者信息。)

\citet[12]{zhuyixuan1985},
\cite{yejianying1946},
\cite{gwywgzjj1958},\cite{shenkuogwywgzjj1070},\cite{jiyun1800},
\citet{liujiang2004},\citet{wanjingkun1996},\citet{dai1983}