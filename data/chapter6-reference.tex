% !Mode:: "TeX:UTF-8"
\chapter{论文的参考文献}
\section{BibTeX简介}
1985年,Oren Patashnik和Lamport开发了BibTeX,其详细使用参阅Nicolas Markey的
《Tame the BeaST: The B to X of BibTeX》。

BibTeX把参考文献的数据放在一个.bib文件中,显示格式放在.bst文件中,普通用户一般不需要改动.bst文件,
只需要维护.bib参考文献数据库即可。

一个.bib文件可以包含多个参考文献条目(entry),每个条目有类型、关键字,以及题目、作者、年份等字段。
常用条目类型有article、book、conference、manual等。每种类型都有一些自己的规定字段和可选字段,字段间
用逗号分隔。数据库中每个条目的关键字(Bibtexkey)要保持唯一,因为引用时需要用到此字段作为此条目的标识。

前文中提到含有交叉引用的文档需要编译两遍(当然也可能没提到)。含有参考文献的文档则更“麻烦”,它需要
依次执行latex、bibtex、latex、latex四次编译操作。
\begin{enumerate}
\item 第一遍latex只把条目的关键字写到中间文件.aux文件中去。
\item bibtex根据.aux、.bib、.bst生成一个.bbl文件,即参考文献列表,它的内容就是thebibliography环境
和一些\verb|\bibetm| 命令。
\item 第二遍latex把交叉引用写到.aux中去。
\item 第三遍latex则在正文中正确地显示引用。
\end{enumerate}

\section{使用JabRef来管理参考文献}
\subsection{JabRef简介}
%可参考网页 http://blog.lehu.shu.edu.cn/shuishousong/A220047.html
\subsection{如何管理参考文献}
\section{参考文献条目要求}
\section{参考文献示例}
