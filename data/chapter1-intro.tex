% !Mode:: "TeX:UTF-8"
\chapter{简介}

\section{项目说明}

欢迎使用北京航空航天大学毕业设计论文毕业设计论文\LaTeX{}模板,
本模板由北航开源俱乐部(BHOSC)维护,根据北京航空航天大学教务处的
本科生毕业设计论文要求和研究生毕业设计论文要求来编写的。

目前本模板支持本科、(学术/专业)硕士和博士研究生毕业设计论文要求规范。

本模板在编写过程中尽可能满足学校要求,但是由于原始规范主要针对Word。
和\LaTeX{}之间不可避免的差异加之编写者的水平限制,本模板很难做到完全一致。
我们十分欢迎北航的\LaTeX{}爱好者/专家参与到本模板的完善工作中,
希望本模板能够对各位同学的论文撰写工作提供便利,
感谢您对我们工作的信任以及任何可能的反馈和贡献。
如果您对开发和完善本模板\,BUAAthesis.cls\,有兴趣,
或者有任何想法和建议,请与我们联系!

{\heiti 注意:}本模板在尽可能满足学校要求的同时,在细节处理上,
倾向于遵从\LaTeX{}排版规范,避免使用奇怪的宏包和编写者认为不规范的设置。
所以难免和学校提供的基于Word的样张存在细微差异,请谨慎使用!

\section{相关信息}

\subsection{模板维护及联系方式}
\begin{tabular}{ll}
    \multicolumn{2}{l}{北航开源俱乐部 BeiHang OpenSource Club (BHOSC)} \\
    GoogleGroup & \url{https://groups.google.com/d/forum/BHOSC/} \\
    Github      & \url{https://github.com/BHOSC/} \\
    IRC         & \#beihang-osc @ FreeNode
\end{tabular}

\subsection{代码托管及相关页面}
\begin{itemize}
    \item 毕业设计论文模板代码
    \item[] \url{https://github.com/BHOSC/BUAAthesis/}
    % TODO(huxuan): Others pages related to BUAAthesis
    % \item 软件学院本科毕设答辩演示模板
    % \item[] \url{https://github.com/huxuan/latex\_buaasoft\_bachelor\_slide}
    % \item 研究生毕设综述报告和开题报告模板
    % \item[] \url{https://github.com/JosephPeng/ZongshuKaiti\}
\end{itemize}

\subsection{贡献者}
\begin{tabularx}{\textwidth}{@{\hspace{2em}}ll}
    \href{https://github.com/JosephPeng/}{Joseph \footnote{目前的维护者}} &
    \href{mailto:mrpeng000@gmail.com}{mrpeng000@gmail.com} \\
    \href{http://huxuan.org/}{huxuan \textsuperscript{1}} &
    \href{mailto:i@huxuan.org}{i@huxuan.org} \\
\end{tabularx}

\subsection{项目协议}
本项目主要遵从以下两套协议
\begin{itemize}
    \item \href{http://www.gnu.org/licenses/gpl.txt}
        {GNU General Public License (GPLv3)}
    \item \href{http://www.latex-project.org/lppl.txt}
        {\LaTeX{} Project Public License (LPPL)}
\end{itemize}
使用前请认真阅读相关协议,详情请见项目代码根目录下的 LICENSE 文件

\section{免责声明}
本模板为编写者依据北京航空航天大学研究生院及教务处出台的
《\href{http://graduate.buaa.edu.cn/ch/u/cms/www/201210/25173751avdq.doc}{北京航空
航天大学研究生撰写学位论文规定(2009年7月修订)}》(下文称“研究生毕业论文格式要求”)和
《\href{http://jiaowu.buaa.edu.cn:8080/edu/viewNewsList.do?newsid=32&queryType=5}{本科生
毕业设计(论文)撰写规范及要求}》(下文答“本科生毕业论文格式要求”)编写而成,
旨在方便北京航空航天大学毕业生撰写学位论文使用。

如前所述,本模板为北航开源俱乐部\LaTeX{}爱好者依据学校的要求规范编写,
研究生院及教务处只提供毕业论文的写作规范,目前并未提供官方\LaTeX{}模板,
也未授权第三方模板为官方模板,故此模板仅为论文规范的参考实现,
不保证格式能完全满足审查老师要求。任何由于使用本模板而引起的论文格式等问题,
以及造成的可能后果,均与本模板编写者无关。

任何组织或个人以本模板为基础进行修改、扩展而生成新模板,请严格遵守相关协议。
由于违反协议而引起的任何纠纷争端均与本模板编写者无关。

\section{本模板与教务处、研究生院所规定的论文格式要求的一些差异}
\subsection{本科毕业论文}
\begin{description}
\item[参考文献] \textbf{本科生毕业论文格式要求}中对参考文献著录规范要求
以GB7714-87《文后参考文献著录格式》为标准,但其给出参考文献的编排示例为
GB7714-2005《文后参考文献著录格式》标准。借鉴\textbf{研究生毕业论文格式要求}中的规范,
本模板亦使用GB7714-2005作为参考文献著录标准。
%其他有待发现及补充
\end{description}

\subsection{研究生毕业论文}
\begin{description}
\item[封面] \textbf{研究生毕业论文格式要求}提供的封面样张中,对于具体间隔的要求并未
硬性规定。本模板参照诸多已提交的硕博士毕业论文进行修正补偿,但在某些项的垂直间隔
的细节方面与\textbf{研究生毕业论文格式要求}略有差别。

\item[页码] 编者未能正确领会\textbf{研究生毕业论文格式要求}中“摘要、目录、图表清单、
主要符号表用{\heiti 五号 Tmise New Roman 体}编连续码”的含义,目前本模板将摘要部分
到主要符号表使用“五号小写罗马数字”进行连续编码。

\item[目录] \textbf{研究生毕业论文格式要求}中对目录格式设置,推荐两种样式,具体见
\textbf{研究生毕业论文格式要求}中附件11。本模板仅支持第二种样式(即附件11中的括号所推荐样式)。

\item[附注] \textbf{研究生毕业论文格式要求}中对图、表的附注,须使用{\heiti 5号宋体}写
在图、表的下方。目前本模板暂未能满足此要求。

\item[参考文献] 对参考文献条目的排列顺序,\textbf{研究生毕业论文格式要求}可“根据正文
中首次引用出现的先后序递增”,或者“按第一作者群的英文字母或拼音字母的英文顺序递增”。本
模板仅支持第一种,即根据正文中首次引用出现的先后序递增。
%其他待发现和补充
\end{description}

\section{版本历史}
\begin{itemize}
	\item[1.1] 2012/12/21 修复大部分细节方面的小错误,完善部分文档工作。
    \item[1.0] 2012/07/24 已完成大体功能,说明文档和细节方面还有待完善。
    % “a.b”为版本号,b为奇数时为内测版本,为偶数时为发行版。
\end{itemize}
