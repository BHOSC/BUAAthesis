\chapter{简介}
	本论文模板为北航开源俱乐部若干\LaTeX{}爱好者根据北京航空航天大学教务处给出的本科生毕业设计论文要求和研究生毕业设计论文要求来编写的。\par
	目前模板支持北航本科毕业设计论文要求规范和(学术/专业)硕士/博士研究生毕业设计论文要求规范,对其格式的后期维护也将进一步的开发和完善。\par
	模板的格式尽量满足北京航空航天大学研究生院和教务处的要求,当然,要求规范中有些地方比较模棱两可,模板编写者的水平有限,不免有些错漏,我们十分欢迎北航的\LaTeX{}爱好者一起参加和完善此模板,为各位奋斗在科研一线的同学们节省不必要的时间,提高北航的论文水平,如果您对开发和完善本模板\,BUAAthesis.cls\,有兴趣,或有任何想法和建议,请与我们联系。谢谢!\par
	\textcolor{red}{\heiti 注意:本模板在编写过程中虽按照上述论文要求规范编写并满足要求,但在某些具体细节上,偏向于\LaTeX{}的排版美化规定,与学校教务处或研究生院给出的样张存在一些细微差异,具体请见{\bf 第X章}以及给出的示例文档酌情使用!}
    \section{版权声明}
版权所有\copyright :北航开源俱乐部(BHOSC)~~---~~\href{mailto:beihang-open-source-club@googlegroups.com}{beihang-open-source-club@googlegroups.com}
	\begin{tabular}{lll}
模板主要贡献者 & Joseph\quad & (\href{mailto:mrpeng000@gmail.com}{mrpeng000@gmail.com})\\
		 ~ & Huxuan & (\href{mailto:huxuan8218528@gmail.com}{huxuan8218528@gmail.com})\\
		~~ & 其他的人自己来添加吧 & (\href{mailto:xxx@gmail.com}{xxx@gmail.com})\\   
	\end{tabular}
	\newline
\qquad 本论文模板为开源自由软件,你可以自由软件基金会发布的《GNU通用公共许可证条款(第三版)》来修改和重新发布这一程序,或者(根据您的选择)来使用任何更新的版本。当然,我们更希望你能与我们合作,一起来完善本模板。
	发布这一模板是希望它能够被用户使用,但没有任何担保,甚至没有适合特定目的的隐含的担保。更详细的情况请参阅《GNU通用公共许可证》\footnote{\url{http://www.gnu.org/licenses/gpl.html}}。

	\section{免责声明}
	本模板的发布遵循\LaTeX{} Project Public License,使用前请认真阅读协议内容。\par
	本模板为作者依据北京航空航天大学研究生院及北京航空航天大学教务处出台的《北京航空航天大学研究生撰写学位论文规定(2009年7月修订)》和《本科生毕业设计(论文)撰写规范及要求》编写而成,旨在为北京航空航天大学毕业生撰写学位论文使用。\par
	正如前面所述,本模板为北航开源俱乐部若干\LaTeX{}爱好者依据研究生院和教务处的要求规范而编写的,研究生院及教务处只提供毕业论文写作规范,不提供官方模板,也不授权第三方模板为官方模板,故此模板仅为论文规范的参考实现,不保证格式能完全满足审查老师要求。任何由于使用本模板而引起的论文格式审查等问题均与本模板作者无关。\par
	任何个人或组织以本模板为基础进行修改、扩展而生成新的专用模板,请严格遵守\LaTeX{} Project Public License 协议。由于违反协议而引起的任何纠纷争端均与本模板作者无关。
	\section{版本历史}
	\begin{itemize}
	\item[*]{\bf{1.0}}\quad 2012/07 \quad 第一个完整版本,包含对研究生毕业论文、本科生毕业设计论文支持。\\
	\item[*]{\bf{0.9}}\quad 2012/06 \quad 编写用户手册。
	%“a.b”为版本号,b为奇数时为内测版本,为偶数时为发行版。好像ubuntu的版本号就是这样的来着?
	\end{itemize}
	%除了itemize环境,还可以使用enumerate环境,如下所示。
	%\begin{enumerate}[(1)]
	%\item This is the first item
	%\end{enumerate}
	