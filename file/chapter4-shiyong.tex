% !Mode:: "TeX:UTF-8"
\chapter{使用说明}
    \section{基本范例}
    \begin{table}[!ht]
    \begin{center}
        \begin{tabular}{|c|c|}
        \hline
        本科生论文基本结构 & 研究生论文基本结构\\\hline\hline
        封面 & 封面(中、英文)\\
        扉页 & 题名页、独创性声明和使用授权书\\
        中英文摘要 & 中英文摘要\\
        目录 & 目录\\
        正文 & 图表清单及主要符号表(根据情况可省略)\\
        致谢 & 主体部分\\
        参考文献 & 参考文献\\
        附录 & 附录\\
        ~~ & 攻读硕士/博士期间取得的研究\slash 学术成果\\
        ~~ & 致谢\\
        ~~ & 作者简介\\
        \hline
        \end{tabular}
        \end{center}
    \end{table}
\qquad 本科生论文结构推荐按如下的代码形式来组织整个论文。具体各个参数及部分请见第xx节。\par
\begin{lstlisting}[language={LaTeX}]
\documentclass[bachelor,oneside,openany]{buaathesis}

%------        论文信息            -------
\thesisauthor{ 张三}{ZhangSan}
\thesistitle{ 论文标题}{English Title}
% 其他的信息请自行按照格式补充

\begin{document}
\maketitle
% !Mode:: "TeX:UTF-8"
\begin{cabstract}
这里是中文摘要部分。加长到一行点来看看效果是怎样的,所以这句话其实只是我用来
测试用的,觉得不爽的话就直接删除吧,哇哈哈~\par
这里是另一段的开始。
\end{cabstract}
\begin{eabstract}
Here is the Abstract in English. And this is a test sentence, just for
a test to see how the buaathesis works. You can just ignore this.\par
This is another pargraph.
\end{eabstract}

\tableofcontents

\mainmatter
\include{file/chapter1}
\include{file/chapter2}
\include{file/chapter3}
\include{file/chapter4}
\clearpage
\chapter*{ 致谢}
\bibliography{file/bibs}
\addcontentsline{toc}{chapter}{ 参考文献}

\appendix
\chapter{ 附录一}
\chapter{ 附录二}
\end{document}

\end{lstlisting}
\qquad 研究生则推荐使用如下的代码形式来组织论文。\par
\begin{lstlisting}[language={LaTeX}]
\documentclass[master]{buaathesis}

%------        论文信息            -------
\thesisauthor{ 张三}{ZhangSan}
\thesistitle{ 论文标题}{English Title}
% 其他的信息请自行按照格式补充

\begin{document}
\maketitle
% !Mode:: "TeX:UTF-8"
\begin{cabstract}
这里是中文摘要部分。加长到一行点来看看效果是怎样的,所以这句话其实只是我用来
测试用的,觉得不爽的话就直接删除吧,哇哈哈~\par
这里是另一段的开始。
\end{cabstract}
\begin{eabstract}
Here is the Abstract in English. And this is a test sentence, just for
a test to see how the buaathesis works. You can just ignore this.\par
This is another pargraph.
\end{eabstract}

\tableofcontents
\listoffigures
\listoftables

\mainmatter
\include{file/chapter1}
\include{file/chapter2}
\include{file/chapter3}
\include{file/chapter4}

\cleardoublepage
\bibliography{file/bibs}
\addcontentsline{toc}{chapter}{ 参考文献}
\cleardoublepage

\appendix
\chapter{ 附录一}
\chapter{ 附录二}

\backmatter
\chapter{ 攻读博士学位期间取得的学术成果}
\chapter{ 致谢}
\chapter{ 作者简介}

\end{document}
\end{lstlisting}
    
    \section{模板选项}
        \subsection{学位选项}
        \begin{description}
            \item{bachelor} 学士学位;
            \item{master} 学术硕士学位(默认);
            \item{engineer} 专业硕士学位;
            \item{doctor} 博士学位。
        \end{description}

        \subsection{其他选项}
        \begin{description}
            \item{oneside\slash twoside} 单面\slash 双面(默认)打印;
            \item{openany\slash openright} 新的章节在任何页面开始\slash 新的章节从奇数页开始(默认);
            \item{nocolor} 所有链接文字、关键字均为黑色并使用框线表示(锚点为红色方框,超链接为淡蓝色方框,打印时方框将不会被打印),默认关闭。
        \end{description}

    \section{封面及正文前的一些设置}
        \subsection{封面}
        \subsection{英文封面(研究生)}
        \subsection{题名页(研究生)}
        \subsection{中英文摘要}
        \subsection{目录}
        \subsection{图表目录及符号列表(研究生)}
    \section{正文}
        \subsection{章节}
        \subsection{参考文献}
    \section{正文之后的内容}
        \subsection{附录}
        \subsection{攻读硕士\slash 博士期间所取得的研究\slash 学术成果(研究生)}
        \subsection{致谢}
        \subsection{作者简介(研究生)}